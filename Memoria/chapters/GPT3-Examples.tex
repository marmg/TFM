%%%%%%%%%%%%%%%%%%%%%%%%%%%%%%%%%%%%%%%%%%%%%%%%%%%
%%% CONCLUSIONES Y TRABAJO FUTURO
%%%%%%%%%%%%%%%%%%%%%%%%%%%%%%%%%%%%%%%%%%%%%%%%%%%

\chapter{Examples feed to GPT-3}
\label{ch:GPT3-Examples}
\noindent As said before, to force \emph{GPT-3} to output an expected result, it has to be feed with some examples of the desired output to know what has to generate. In order to force it to generate explanations of the prediction, first a few examples of the test set has been formatted and added an explanation as desired. These are the full examples feed to \emph{GPT-3}:

\begin{passage}[Article of GPT Example 1]{art:gpt1}
 It was the golden season. I could see the yellow leaves dancing in the cool wind. I felt lonely and life is uninteresting. But one day, the sound of a violin came into my ears. I was so surprised that I ran out to see where it was from. A young girl, standing in the wind, was lost in playing her violin. \\
I had never seen her before. The music was so wonderful that I forgot who I was. \\
Leaves were still falling. Every day she played the violin in the same place and I was the only listener. It seemed that I no longer felt lonely and life became interesting. We didn't know each other, but I thought we were already good friends.
One day, when I was listening, the sound suddenly stopped. The girl came over to me. \\
``You must like violin.'' she said. \\
``Yes. And you play very well. Why did you stop?'' I asked. \\
Suddenly, a sad expression appeared on her face and I could feel something unusual. \\
``I came here to see my grandmother, but now I must leave. I once played very badly. It is your listening every day that has \_ me.'' she said. \\
``In fact, it is your music that has given me those meaningful days.'' I answered. ``Let us be friends.'' \\
The girl smiled and I smiled. \\
I never heard her play again in my life. Only thick leaves were left behind. But I will always remember the girl. She is like a dream; so short, so bright that it makes life beautiful. \\
There are many kinds of friends. Some are always with you, but don't understand you. Some say only a few words to you, but are close to you. I shall always think of those golden days and the girl with the violin. She will always bring back the friendship between us. I know she will always be my best friend.
\end{passage}
Q: ``\emph{What's the best title for the passage?}''\\
Options: 
\begin{itemize}
 \item A: A Musical Girl
 \item B: Wonderful Music
 \item C: A Special Friend in a Special Autumn
 \item D: How to Be Friends
\end{itemize}
Answer: C. \\
Explanation: It says that it was the golden season and autumn is known as the golden season and it speaks about friendship not just music.

\begin{passage}[Article of GPT Example 2]{art:gpt2}
 Is it important to have breakfast every day? A short time ago, a test was given in the United States. People of different ages, from 12 to 83, were asked to have a test. During the test, these people were given all kinds of breakfast, and sometimes they got no breakfast at all. Scientists wanted to see how well their bodies worked after eating different kinds of breakfast. \\
The results show that if a person eats a right breakfast, he or she will work better than if he or she has no breakfast. If a student has fruit, eggs, bread and milk before going to school, he or she will learn more quickly and listen more carefully in class. Some people think it will help you lose weight if you have no breakfast. But the result is opposite to what they think. This is because people become so hungry at noon that they eat too much for lunch. They will gain weight instead of losing it.
\end{passage}
Q: ``\emph{According to the passage, what will happen to you if you don't have any breakfast?}''\\
Options: 
\begin{itemize}
 \item A: To be healthier.
 \item B: To work better.
 \item C: To gain weight.
 \item D: To fail the test.
\end{itemize}
Answer: C. \\
Explanation: Because it says that people become so hungry at noon that they eat too much for lunch.

\begin{passage}[Article of GPT Example 3]{art:gpt3}
 It's the second time for me to come to Beijing. There are many places of interest in Beijing, such as the Summer Palace, the Great Wall, etc. What's more, I think great changes have taken place in Beijing. People's living conditions have improved a lot. Their life is very happy. Almost everyone has a big smile on the face. People in Beijing are in high spirits and hard-working. Children can receive a good education. \\
But in the past, some children didn't have enough money to go to school. They often worked for cruel bosses. The bosses didn't give them enough food. I feel sorry for them. Today people have already lived in tall building, worn beautiful clothes and so on. Life has changed greatly.
\end{passage}
Q: ``\emph{How many times has the writer been to Beijing?}''\\
Options: 
\begin{itemize}
 \item A: Once.
 \item B: Twice.
 \item C: Three time.
 \item D: Four times.
\end{itemize}
Answer: B. \\
Explanation: Becuase it says that is the second time that he goes to Beijing.

\begin{passage}[Article of GPT Example 4]{art:gpt4}
 Several years ago,a television reporter was talking to three of the most important people. One was a very rich banker,another owned one of the largest companies in the world,and the third owned many buildings in the center of New York. \\
The reporter was talking to them about being important. ``How do we know if someone is really important?'' the reporter asked the banker. \\
The banker thought for a few moments and then said, ``I think anybody who is invited to the White House to meet the President of the United States is really important.'' \\
The reporter then turned to the owner of the very large company. ``Do you agree with that?'' she asked. \\
The man shook his head, ``No. I think the President invites a lot of people to the White House. You'd only be important if while you were visiting the President, there was a telephone call from the president of another country,and the President of the US said he was too busy to answer it.'' \\
The reporter turned to the third man. ``Do you think so?'' \\
``No, I don't,'' he said. ``I don't think that makes the visitor important. That makes the President important. '' \\
``Then what would make the visitor important?'' the reporter and the other two men asked. \\
``Oh, I think if the visitor to the White House was talking to the President and the phone rang, and the President picked up the receiver, listened and then said, 'It's for you.' '' \\
\end{passage}
Q: ``\emph{This story happened in   \_  .}''\\
Options: 
\begin{itemize}
 \item A: England.
 \item B: America.
 \item C: Japan.
 \item D: Australia.
\end{itemize}
Answer: B. \\
Explanation: It talks about New York, the United states and the White House that are in America.

\begin{passage}[Article of GPT Example 5]{art:gpt5}
 Dear John, Thank you very much for your letter. I am glad that you enjoyed your holiday with me. We enjoyed having you and your sister here. We hope that you will both be able to come again next year. Perhaps you'll be able to stay longer next time you come. A week is not really long enough, is it? If your school has a five-week holiday next year, perhaps  you'll be able to stay with us for two or three weeks. \\
We have been back at school three weeks now. It feels like three months! I expect  that you are both working very hard now that you are in Grade One. I shall have to work hard next year when I am in Grade One. Tom and Ann won't be in Grade One until 2011.
They went for a picnic yesterday but I didn't go with them because I cut my foot and I couldn't walk very well. They went to an island and enjoyed themselves. Do you still remember the island? That's where all five of us spent the last day of our holiday.
Tom, Ann and I send our best wishes to Betty and you. We hope to see you soon. \\
Yours sincerely, \\
Michael
\end{passage}
Q: ``\emph{From the words of  ``It feels like three months!'' we know that  \_  .}''\\
Options: 
\begin{itemize} 
 \item A: Michael's teacher is very strict with the students.
 \item B: Michael is pleased with his school report.
 \item C: Michael has no interest in learning.
 \item D: Michael works very hard at his studies.
\end{itemize}
Answer: C. \\
Explanation: When something feels to go slower is because we dont like it. If we do like it seems to go faster.
