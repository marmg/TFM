%%%%%%%%%%%%%%%%%%%%%%%%%%%%%%%%%%%%%%%%%%%%%%
%%% INTRODUCCION
%%%%%%%%%%%%%%%%%%%%%%%%%%%%%%%%%%%%%%%%%%%%%%

\chapter{Introduction}
\fancyhead[RE]{\textsc{Chapter} \thechapter. Introduction}
\label{ch:Introduction}
\pagenumbering{arabic}

\section{Motivation}
\label{sec:Motivation}
\noindent

\section{Proposal and goals}
\label{sec:ProposalAndGoals}
\noindent

\section{Document structure}
\label{sec:DocumentStructure}
\noindent

\begin{description}
\item \textbf{Chapter 1. Introduction.} In this chapter the main motives for the realization of this essay, the proposal and the goals to achieve are presented, as well as the document structure.
\item \textbf{Chapter 2. State of the art.} The discipline and the task are reviewed in this chapter, including a brief introduction of the explainability, the language models and the question answering task and the state of the art of this areas. At the end of this chapter, the state of the art of explainability applied to Natural Language Processing tasks is reviewed.
\item \textbf{Chapter 3. Methodology.}  The different methods and algorithms used to solve the problem are introduced and studied in this chapter.
\item \textbf{Chapter 4. Results.} After knowing the different methods and algorithms being used, the experiments with their results are presented in this chapter.
\item \textbf{Chapter 5. Conclusions and future work.} The conclusions of the review of the state of the art in addition to the conclusion of the usage of the different methods and to the future work that could be done after this project.
\end{description} 